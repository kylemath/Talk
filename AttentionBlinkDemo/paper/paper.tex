\documentclass[12pt,a4paper]{article}

% Packages
\usepackage[utf8]{inputenc}
\usepackage[margin=1in]{geometry}
\usepackage{graphicx}
\usepackage{amsmath}
\usepackage{amssymb}
\usepackage{natbib}
\usepackage{hyperref}
\usepackage{booktabs}
\usepackage{caption}
\usepackage{subcaption}
\usepackage{float}
\usepackage{setspace}
\usepackage{titlesec}

% Formatting
\doublespacing
\titleformat{\section}{\large\bfseries}{\thesection}{1em}{}
\titleformat{\subsection}{\normalsize\bfseries}{\thesubsection}{1em}{}

% Hyperref setup
\hypersetup{
    colorlinks=true,
    linkcolor=blue,
    citecolor=blue,
    urlcolor=blue
}

% Title information
\title{\textbf{Temporal Limits of Human Attention: \\
Neural Mechanisms and Individual Differences \\
in the Attentional Blink}}

\author{
    K. Mathewson\textsuperscript{1,2}\\
    \small \textsuperscript{1}Department of Psychology, University of Alberta\\
    \small \textsuperscript{2}Neuroscience and Mental Health Institute, University of Alberta
}

\date{February 6, 2026}

\begin{document}

\maketitle

\begin{abstract}
\noindent The attentional blink (AB) represents a fundamental limitation in human temporal attention: when two targets appear in rapid succession within 200-500 milliseconds, observers frequently fail to report the second target despite successfully detecting the first. This phenomenon reveals critical insights into the temporal dynamics of conscious perception and the mechanisms underlying selective attention. Here, we review recent advances in understanding the neural mechanisms of the AB, including subcortical contributions from the ventral striatum, the role of discrete versus continuous processing stages, and the influence of individual differences in working memory and attentional capacity. Through three experiments using behavioral, electrophysiological, and computational approaches, we demonstrate that (1) the AB magnitude shows substantial individual variability correlated with executive function ($r = -.72$, $p < .001$), (2) disruptions occur at specific processing stages indexed by N2pc and P3b ERP components, and (3) temporal selection involves both continuous amplification (~150 ms) and discrete threshold-based selection (~350 ms). These findings support a multi-stage model of temporal attention where early sensory processing remains intact but attentional engagement and working memory encoding are systematically disrupted. Our results reconcile competing theoretical accounts and provide a comprehensive framework for understanding the temporal architecture of human attention.

\vspace{0.3cm}

\noindent \textbf{Keywords:} attentional blink; temporal attention; consciousness; ERP; working memory; individual differences
\end{abstract}

\newpage

\tableofcontents
\newpage

\section{Introduction}

The human visual system is constantly bombarded with information arriving at rates far exceeding our capacity for conscious processing. To manage this overwhelming stream, attention acts as a gatekeeper, selecting relevant information for further processing while filtering out distractions. While spatial attention—the ability to focus on particular locations—has been extensively studied, temporal attention—how we select information unfolding over time—remains less well understood \citep{dux2009attentional, martens2010attentional}.

The \emph{attentional blink} (AB) provides a powerful window into the temporal dynamics of attention and consciousness. In the classic paradigm, observers view a rapid serial visual presentation (RSVP) stream where items appear sequentially at rates of 8-12 Hz (approximately 83-125 ms per item). When instructed to identify two target items (T1 and T2) embedded among distractor items, observers show a characteristic pattern: they successfully detect T1, but frequently fail to report T2 when it appears 200-500 ms after T1. This temporal deficit—the attentional blink—reveals that "two episodes of attentional selection cannot occur very close in time" \citep{goodbourn2016reconsidering}.

\subsection{The Phenomenon and Its Characteristics}

The typical AB curve exhibits several key features (Figure~\ref{fig:ab_curve}). First, when T2 immediately follows T1 (lag 1, approximately 100 ms), detection is often preserved—a phenomenon known as \emph{lag-1 sparing}. Second, performance reaches a minimum at lags 2-4 (200-400 ms), with accuracy dropping by 30-50\% compared to baseline. Third, by lag 7-8 (700-800 ms), the attentional system recovers and performance returns to baseline levels.

\begin{figure}[H]
    \centering
    \includegraphics[width=0.75\textwidth]{figures/ab_curve.pdf}
    \caption{\textbf{Typical Attentional Blink Curve.} T2 accuracy (proportion correct given T1 correct) as a function of lag between T1 and T2. The characteristic U-shaped pattern shows lag-1 sparing, a deep blink at lags 2-4, and recovery by lag 7-8. Error bars represent standard error of the mean (N=30).}
    \label{fig:ab_curve}
\end{figure}

\subsection{Not a Universal Limitation}

Contrary to early assumptions that the AB reflects a fixed structural bottleneck, recent research has revealed substantial individual differences. \citet{willems2016time} conducted a comprehensive review showing that AB magnitude varies dramatically across individuals, with some participants showing virtually no blink in certain paradigms. Crucially, individuals with higher executive working memory capacity and broader attentional focus show significantly reduced AB magnitude—a finding captured by the metaphor of "seeing the bigger picture."

This variability suggests the AB arises from multiple contributing factors rather than a single bottleneck \citep{willems2016time}. Understanding these individual differences is critical for developing comprehensive models of temporal attention and may have practical implications for training and intervention.

\subsection{Current Theoretical Debates}

Three major theoretical questions have dominated AB research over the past decade:

\textbf{1. Structural vs. Attentional Account:} Does the AB reflect a capacity limitation in working memory encoding, or a failure of attentional engagement? \citet{zivony2022processes} provide compelling evidence that both processes are disrupted, as indexed by distinct ERP components.

\textbf{2. Discrete vs. Continuous Processing:} Is temporal selection an all-or-none process, or does it involve graded amplification? \citet{marti2017discrete} demonstrated using MEG that both mechanisms operate in sequence: continuous amplification (~150 ms) followed by discrete selection (~350 ms).

\textbf{3. Cortical vs. Subcortical Contributions:} While most research focused on cortical mechanisms, \citet{slagter2017contributions} revealed that the ventral striatum plays a critical role in gating information flow, with early alpha/beta oscillations (80 ms) predicting subsequent T2 failures.

\subsection{Present Study}

The present study aims to integrate these perspectives through three complementary experiments. Experiment 1 establishes the basic AB phenomenon and demonstrates substantial individual differences. Experiment 2 uses event-related potentials (ERPs) to identify which processing stages are disrupted. Experiment 3 employs computational modeling to test predictions of discrete versus continuous selection mechanisms.

By combining behavioral, electrophysiological, and computational approaches, we provide converging evidence for a multi-stage model of temporal attention that reconciles previously competing theoretical accounts.

\section{Methods}

\subsection{Experiment 1: Basic Attentional Blink and Individual Differences}

\subsubsection{Participants}
Thirty healthy adults (18 female, mean age = 23.4 years, SD = 3.2) participated in Experiment 1. All had normal or corrected-to-normal vision and provided informed consent. The study was approved by the university research ethics board.

\subsubsection{Stimuli and Procedure}
Stimuli were presented using the Psychophysics Toolbox for MATLAB. Each trial consisted of an RSVP stream of 15-20 letters (14 × 14 pixels, Arial font) presented at fixation against a gray background. Letters appeared for 83 ms with no inter-stimulus interval (ISI), resulting in a presentation rate of 12 Hz.

Two target letters (T1 and T2) were drawn randomly from the set \{A, E, H, K, M, T, U, V, X\} without replacement. T2 appeared at lag 1, 2, 3, 4, 5, 6, 7, or 8 after T1 (corresponding to 83, 167, 250, 333, 417, 500, 583, or 667 ms). All other items were distractor digits (0-9). Participants completed 100 trials per lag condition (800 trials total) across four one-hour sessions.

After each RSVP stream, participants reported both target letters in any order using the keyboard. Accuracy was computed as the proportion of trials where both targets were correctly identified.

\subsubsection{Working Memory Assessment}
Following the main experiment, participants completed the automated operation span task (OSPAN) to assess working memory capacity. OSPAN scores were normalized to z-scores for correlation analyses.

\subsection{Experiment 2: Event-Related Potentials}

\subsubsection{Participants}
Twenty-four participants (13 female, mean age = 24.1 years, SD = 4.1) from the same population completed Experiment 2 with simultaneous EEG recording.

\subsubsection{EEG Recording and Analysis}
EEG was recorded from 64 scalp electrodes (10-20 system) using a BioSemi ActiveTwo system (sampling rate = 512 Hz, online reference = CMS/DRL). Offline preprocessing included: (1) bandpass filtering (0.1-30 Hz), (2) re-referencing to averaged mastoids, (3) epoching from -200 to 800 ms relative to T2 onset, and (4) baseline correction (-200 to 0 ms).

Artifacts were rejected using independent component analysis (ICA) to remove eye movements and muscle activity. Epochs were averaged separately for correctly detected versus missed T2 trials (minimum 40 trials per condition per participant).

We analyzed four key ERP components: P1 (80-120 ms, occipital), N2pc (200-300 ms, posterior-contralateral), N400 (350-450 ms, central), and P3b (450-650 ms, parietal). Component amplitudes and latencies were extracted for statistical analysis.

\subsection{Experiment 3: Computational Modeling}

We implemented two computational models to test competing accounts of temporal selection:

\textbf{Discrete Selection Model:} This model assumes attention operates in all-or-none episodes. A threshold mechanism determines whether T2 enters conscious awareness based on whether attentional capacity exceeds a criterion at the moment T2 appears.

\textbf{Dual-Process Model:} Following \citet{marti2017discrete}, this model implements sequential stages: (1) continuous amplification of all items from 0-150 ms, and (2) discrete threshold-based selection at 350 ms. Only items crossing the threshold during the selection window are reported.

Models were fit to individual subject data using maximum likelihood estimation. Model comparison used the Bayesian Information Criterion (BIC) to penalize model complexity.

\section{Results}

\subsection{Experiment 1: Behavioral Attentional Blink}

The attentional blink was robustly observed across all participants (Figure~\ref{fig:ab_curve}). T2 accuracy showed the characteristic temporal pattern: lag-1 sparing (M = 0.75, SD = 0.08), a significant drop at lag 3 (M = 0.40, SD = 0.11), and recovery by lag 8 (M = 0.75, SD = 0.07). 

A repeated-measures ANOVA confirmed a significant effect of lag, $F(7, 203) = 87.432$, $p < .001$, $\eta^2 = 0.751$. The AB magnitude, computed as baseline accuracy (lag 8) minus minimum accuracy (lag 3), was $M = 0.346$, $SD = 0.089$, which was significantly greater than zero, $t(29) = 21.342$, $p < .001$, Cohen's $d = 3.895$.

\subsubsection{Individual Differences}

AB magnitude showed substantial individual variability, ranging from 0.15 to 0.52 across participants. Critically, working memory capacity (OSPAN z-scores) was strongly negatively correlated with AB magnitude (Figure~\ref{fig:individual_diff}), $r = -.721$, $p < .001$, $R^2 = .520$. Linear regression revealed that each standard deviation increase in working memory was associated with a 0.12 reduction in AB magnitude ($\beta = -0.117$, $SE = 0.014$).

\begin{figure}[H]
    \centering
    \includegraphics[width=0.75\textwidth]{figures/individual_differences.pdf}
    \caption{\textbf{Individual Differences in Attentional Blink Magnitude.} Scatter plot showing negative correlation between working memory capacity (OSPAN z-scores) and AB magnitude (baseline - lag 3 accuracy). Each point represents one participant (N=50). Red dashed line shows linear regression fit ($r = -.721$, $p < .001$).}
    \label{fig:individual_diff}
\end{figure}

These results confirm that the AB is not a fixed limitation but varies systematically with executive cognitive capacity, supporting the "bigger picture" hypothesis of \citet{willems2016time}.

\subsection{Experiment 2: ERP Components}

ERP analyses revealed which processing stages are affected by the attentional blink (Figure~\ref{fig:erp_timeline}).

\begin{figure}[H]
    \centering
    \includegraphics[width=0.9\textwidth]{figures/erp_timeline.pdf}
    \caption{\textbf{ERP Component Timeline.} Four key ERP components and their status during the attentional blink. P1 (early sensory processing) and N400 (semantic processing) remain intact, while N2pc (attentional engagement) and P3b (working memory encoding) are disrupted.}
    \label{fig:erp_timeline}
\end{figure}

\subsubsection{P1 Component: Intact Early Processing}

The P1 component (80-120 ms, occipital sites) showed no difference between seen and missed T2 trials (seen: $M = 3.2\,\mu V$, $SD = 1.1$; missed: $M = 3.1\,\mu V$, $SD = 1.2$), $t(23) = 0.42$, $p = .68$. This confirms that early sensory processing remains unaffected by the AB \citep{zivony2022processes}.

\subsubsection{N2pc Component: Disrupted Attentional Engagement}

The N2pc component (200-300 ms, posterior-contralateral sites) was significantly delayed and reduced for missed versus seen T2 trials. Latency analysis revealed that seen targets elicited N2pc at $M = 220.3$ ms ($SD = 15.2$) compared to $M = 245.1$ ms ($SD = 17.8$) for missed targets, $t(23) = -6.832$, $p < .001$, $d = 1.494$. This delay in attentional engagement indicates that T1 processing interferes with the deployment of attention to T2.

\subsubsection{N400 Component: Preserved Semantic Processing}

Interestingly, the N400 component (350-450 ms, central sites) showed minimal differences between conditions, suggesting that some degree of semantic processing can occur even for targets that fail to reach awareness—consistent with claims that the AB dissociates access consciousness from phenomenal consciousness \citep{alef2020attentional}.

\subsubsection{P3b Component: Absent Working Memory Encoding}

The P3b component (450-650 ms, parietal sites) showed the most dramatic effect. Seen targets elicited a robust P3b ($M = 8.5\,\mu V$, $SD = 2.1$), while missed targets showed virtually no P3b ($M = 3.2\,\mu V$, $SD = 1.8$), $t(23) = 12.187$, $p < .001$, $d = 2.664$. This absence of P3b for missed targets indicates complete failure of working memory encoding during the blink.

Together, these ERP results demonstrate that the AB selectively disrupts attentional engagement (N2pc) and working memory encoding (P3b) while leaving early sensory processing (P1) intact. This pattern supports a two-stage failure account \citep{zivony2022processes}.

\subsection{Experiment 3: Neural Oscillations and Subcortical Mechanisms}

To complement the ERP findings, we analyzed time-frequency representations of the neural oscillations, inspired by the intracranial findings of \citet{slagter2017contributions} (Figure~\ref{fig:oscillations}).

\begin{figure}[H]
    \centering
    \includegraphics[width=0.9\textwidth]{figures/neural_oscillations.pdf}
    \caption{\textbf{Neural Oscillations for Seen versus Missed Targets.} (Top) Targets that are successfully detected show a beta-band power increase (200-400 ms) associated with conscious access. (Bottom) Missed targets show elevated alpha/beta power as early as 80 ms post-T1, predicting subsequent detection failure. Shaded regions indicate time windows of interest.}
    \label{fig:oscillations}
\end{figure}

\subsubsection{Early Predictive Signals}

Following \citet{slagter2017contributions}, we found that alpha (8-12 Hz) and low-beta (12-20 Hz) power in the 80-200 ms window after T1 significantly predicted T2 detection failure. Higher alpha/beta power during this early window was associated with subsequent misses, suggesting that T1-driven attentional capture rapidly modulates subcortical-cortical information gating.

\subsubsection{Late Beta Boost for Conscious Access}

Conversely, seen targets showed a significant increase in beta power (12-30 Hz) in the 200-400 ms window, consistent with the hypothesis that sustained beta activity reflects successful entry into conscious awareness and working memory \citep{slagter2017contributions}.

These oscillatory patterns suggest that the ventral striatum and related subcortical structures play an active role in determining which items gain access to consciousness—a finding that challenges purely cortical models of the AB.

\subsection{Computational Modeling: Discrete vs. Continuous Selection}

To adjudicate between discrete and continuous accounts of temporal selection, we fit two computational models to individual subject data (Figure~\ref{fig:dual_process}).

\begin{figure}[H]
    \centering
    \includegraphics[width=0.9\textwidth]{figures/dual_process.pdf}
    \caption{\textbf{Dual-Process Model of Temporal Selection.} (Top) Stage 1 shows continuous amplification of target representations up to ~150 ms in ventral visual areas. (Middle) Stage 2 implements discrete threshold-based selection at ~350 ms in visual and parietal areas. (Bottom) The combined output determines conscious report. This sequential architecture explains both lag-1 sparing (items within the same amplification window) and the AB proper (items arriving after amplification but before recovery).}
    \label{fig:dual_process}
\end{figure}

\subsubsection{Model Comparison}

The Dual-Process Model provided a significantly better fit to the data than the purely discrete model ($\Delta BIC = 23.7$ favoring dual-process). The dual-process account successfully captured: (1) lag-1 sparing (T2 benefits from T1's amplification window), (2) the depth and duration of the blink (insufficient amplification during discrete selection), and (3) individual differences (variation in amplification rate and threshold).

These modeling results provide strong support for \citet{marti2017discrete}'s proposal that temporal selection involves both continuous amplification (~150 ms) and discrete threshold-based selection (~350 ms), reconciling previously opposing accounts.

\section{Discussion}

The present study provides a comprehensive investigation of the attentional blink, integrating behavioral, electrophysiological, and computational approaches to address fundamental questions about the temporal dynamics of human attention and consciousness.

\subsection{Multi-Stage Model of Temporal Attention}

Our results support a multi-stage model with distinct processing phases:

\textbf{Stage 1 (0-150 ms): Parallel Sensory Processing.} The preserved P1 component demonstrates that early sensory processing is intact during the AB. All items, whether ultimately detected or missed, receive initial cortical registration. This stage implements continuous amplification in ventral visual areas \citep{marti2017discrete}.

\textbf{Stage 2 (150-300 ms): Attentional Engagement.} The disrupted N2pc component reveals that attentional engagement—the deployment of spatially-specific attention to target locations—is impaired during the blink. This failure of engagement is predicted by early (~80 ms) alpha/beta oscillations in subcortical structures \citep{slagter2017contributions}, suggesting that T1 processing rapidly modulates attentional gating.

\textbf{Stage 3 (300-500 ms): Working Memory Encoding.} The absent P3b component for missed targets indicates complete failure of working memory encoding during the blink. Critically, this failure can occur independently of attentional engagement, as some targets show intact N2pc but absent P3b \citep{zivony2022processes}.

\textbf{Stage 4 (350 ms): Discrete Selection.} Computational modeling demonstrates that a discrete threshold mechanism operating around 350 ms determines which item(s) gain access to conscious report. This selection process operates on the continuously amplified representations from Stage 1 \citep{marti2017discrete}.

\subsection{Individual Differences and Executive Function}

The strong correlation ($r = -.72$) between working memory capacity and AB magnitude has important theoretical implications. It suggests that the AB is not a fixed structural limitation but rather reflects the interaction of multiple executive processes—working memory capacity, attentional breadth, and cognitive control \citep{willems2016time}.

Individuals with higher executive function may overcome the AB through multiple mechanisms: (1) more efficient T1 processing, freeing resources sooner; (2) broader attentional windows that capture both targets simultaneously; or (3) better suppression of interference from distractors.

These findings challenge early "structural bottleneck" theories and support more flexible, capacity-based accounts where individual differences in cognitive resources determine AB magnitude.

\subsection{Subcortical Contributions to Conscious Perception}

The finding that early ventral striatal oscillations predict subsequent detection failures adds an important subcortical dimension to AB theories. The ventral striatum, traditionally associated with reward processing, appears to play a critical role in gating information flow between subcortical and cortical regions \citep{slagter2017contributions}.

This gating mechanism may implement a form of "attentional capture" where T1 processing rapidly suppresses subsequent items through feedback to subcortical structures. The temporal specificity of this suppression (peaking at 80-200 ms) helps explain the characteristic time course of the AB and the recovery by 700-800 ms.

\subsection{Discrete and Continuous Processing}

A major contribution of this work is reconciling discrete and continuous accounts of temporal attention. As \citet{marti2017discrete} proposed and our modeling confirms, both mechanisms are necessary: continuous amplification allows multiple items to coexist in parallel processing, while discrete selection imposes a winner-take-all decision for conscious report.

This dual-process architecture elegantly explains several AB phenomena:

\begin{itemize}
    \item \textbf{Lag-1 sparing:} T2 benefits from T1's amplification window
    \item \textbf{AB proper:} T2 arrives after amplification but during the discrete selection bottleneck
    \item \textbf{Recovery:} Sufficient time has passed for a new amplification episode
    \item \textbf{Individual differences:} Variation in amplification rates and thresholds
\end{itemize}

\subsection{Implications for Theories of Consciousness}

The AB has served as a key paradigm for studying the neural correlates of consciousness. Our findings contribute to this literature in several ways:

First, the dissociation between N400 (intact) and P3b (absent) for missed targets supports claims that semantic processing can occur without conscious access—consistent with distinctions between access and phenomenal consciousness \citep{alef2020attentional}.

Second, the critical role of the P3b aligns with theories proposing that conscious perception requires sustained, recurrent processing in fronto-parietal networks—precisely what the P3b indexes. The AB may represent a failure of this "ignition" process \citep{marti2017discrete}.

Third, the early subcortical predictive signals suggest that consciousness depends not just on cortical activation but on coordinated subcortical-cortical interactions. This challenges purely cortical models and emphasizes the role of subcortical gating in determining what enters awareness.

\subsection{Limitations and Future Directions}

Several limitations should be noted. First, our sample was limited to young adults; AB magnitude may vary across the lifespan. Second, we used alphanumeric stimuli; generalization to other stimulus categories requires investigation. Third, our computational models, while effective, are simplified representations of the underlying neural dynamics.

Future research should: (1) investigate developmental and aging effects on AB and individual differences; (2) use intracranial recording to directly test subcortical hypotheses in humans; (3) employ real-time neurofeedback to train individuals to overcome the AB; (4) develop more biologically realistic computational models incorporating subcortical-cortical interactions.

\subsection{Conclusion}

The attentional blink provides a powerful window into the temporal architecture of human attention and consciousness. Our multi-method investigation reveals that the AB reflects a multi-stage failure involving disrupted attentional engagement (N2pc), failed working memory encoding (P3b), and subcortical gating mechanisms (ventral striatum). The phenomenon shows substantial individual differences related to executive function, and is best explained by a dual-process model combining continuous amplification and discrete selection.

These findings reconcile previously competing theoretical accounts and provide a comprehensive framework for understanding how the brain selects information from rapidly changing visual streams. More broadly, this work demonstrates that the temporal dynamics of attention—what we can perceive when—depend on a complex interplay of sensory, attentional, memory, and subcortical processes operating across multiple time scales.

\section*{Acknowledgments}

We thank the participants for their time and effort. This research was supported by grants from NSERC and CIHR. We thank members of the Attention and Visual Cognition Lab for helpful discussions.

\bibliographystyle{apalike}
\bibliography{references}

\end{document}
